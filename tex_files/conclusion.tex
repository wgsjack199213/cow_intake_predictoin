\section{小结与下一步工作方向}
\label{futurework}

\subsection{实验结论整理}

xgboost模型对历史数据的拟合能力很强。但是头均采食量预测任务要求模型具有较好的泛化性能(对未见过数据样本的预测能力)。因此实验需要考察拆分训练集、测试集,交叉验证的结果。本工作未划分验证集。未来工作可考虑划分训练集、测试集、验证集。


通过实验我们发现使用单日观测数据对第二天头均采食量进行预测的效果较差,当引入时域信息(用最近几日头均采食量或产奶量的时序特征)后,模型的预测性能能够提升,但提升不够显著。我们在不同实验中得到的最优结果1.0788相对于对照组1.1560有0.077的提升,来自于除基本特征外,引入近4天头均产奶量历史数据1阶小波分解的系数作为特征构建的模型。

以得到最优预测结果的模型为例,通过对各特征的重要性进行分析,可知对xgboost模型预测第二天头均采食量最重要的特征是当天的头均采食量,泌乳天数和THI,其次最近4天头均产奶量时间序列小波分解得到的1阶近似和1阶细节,泌乳期(胎次)的权重最低。


当前基于数据的利用XGBoost模型建模、预测头均采食量的方法制定备草量仍然具有一定的采食不足或剩草过多的风险。为了进一步优化草量投放量,我们可以从多个角度进行探索,包括:饲养员的角度,模型预测能力的角度,草料制备策略的角度和草料投放策略的角度。在探索各个角度的优化策略的同时,不能忽略了相关手段可能带来的额外成本。

\subsection{未来工作方向}

接下来工作主要分为三块:数据集扩充,特征扩充,特征工程。

\para{数据集扩充:}扩大数据集样本量,积累更多数据。数据积累需要时间。如历史上(2017年3月以前)有各牛棚新增的采食量(或剩草量)数据,也可合并进入当前数据集。且在未来如有可能可在牛棚中增加部署传感器,例如温湿度传感器等等。当前工作采用的数据集覆盖时间段为3月至7月,此段期间THI指数整体稳步上升(由稳步上升的温度主导),如能扩大数据集的时间段至覆盖全年,则可以使训练集学习到更多THI趋势(平稳、下降)下头均采食量的变化规律。

\para{特征扩充:}未来工作可以将其他有记录的应激情况(如牛棚疫苗注射记录)纳入分析。如有可能,可另外增加牛的运动量数据(如计步器采集数据)。

\para{特征工程和模型:}尝试各种特征提取、变换方式以提升模型的预测性能,或者在数据集足够大的情况下尝试神经网络模型。