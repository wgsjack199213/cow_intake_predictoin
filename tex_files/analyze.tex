\section{问题分析和形式化描述}
\label{analyze}

\subsection{采食量的影响因素}

泌乳牛自由采食量主要受到内部因素和外部因素影响。部分相关因素列举如下:

\para{内部因素:} 

\begin{itemize}
\item
奶牛品种
\item
奶牛泌乳期(胎次,对应不同的泌乳周期,如初产牛、经产牛)
\item
 奶牛泌乳周(或泌乳天数,对应同一泌乳周期中的不同阶段,如高产、中产、低产、干奶)
\item
奶牛体重
\item
奶牛产奶量(产奶浄能)
\item
奶牛运动量
\item
奶牛身心状态(疾病、情绪等)
\end{itemize}

\para{外部因素:} 
\begin{itemize}
\item
饲料特性
\item
温度湿度(热应激)
\item
其他应激(如疫苗注射,受到惊吓等)
\item
其他环境因素(如较宽的槽位可提升采食量)

\end{itemize}

理想情况下,在预测牛的采食量时,模型应将尽可能多的上述因素纳入考虑。
但实际情况中常面临两个问题:(1)数据类别采集不全,(2)数据量积累较少。
问题(1)会制约模型预测目标值采食量的能力,因为未观测/记录到的因素会对采食量带来模型无法预测的波动。
问题(2)会影响模型的精度,因为通常基于机器学习或者数据挖掘技术构建的模型在训练集越大,模型性能会越好,尤其是一些复杂度高的模型如人工神经网络(Artifical Neural Network)深度学习(Deep Learning)等。
故在本工作中我们会取舍地考虑部分因素。详情请见第\ref{dataset}节。

\subsection{以牛舍为建模单位}

生产环境中,牧场对草料的发放以及对剩草量的统计以牛舍为单位,故我们不以单头牛的采食量为预测目标,而是以牛舍中牛群整体的采食量(或者等价的牛舍中头均采食量)为预测目标。通常同牛舍内牛的品种相同,泌乳期相近或相同,泌乳天数相近或相同,故我们在建模时可忽略牛舍内单头牛之间的差异,仅考虑牛群整体的特性(或等价的头均特性)。

\subsection{问题形式化}

生产环境中,每牛舍每天上报一次草量,对应当天中午、晚上和第二天早上三次投喂草料的总草量。

我们记某牛舍第t天上午上报的头均草量(报草量/牛的数量)为$r_{t+1}$\footnote{用$t+1$而不是$t$表示是为了避免在时间先后上引起歧义,可以理解$r_{t+1}$为给牛分配的吃到第$t+1$天的草量。},
接下来的一天内实际头均采食量为$y_{t+1}$。
我们用向量$X_t$表示第$t$天的输入变量。具体地,$X_t$的每一个元素为一个关于牛的或关于环境的观测数据,如第$t$天牛的产奶量等等。

我们希望得到一个预测模型$f$使得
\begin{equation}
	\hat y_{t+1} = f(X_t, X_{t-1}, \cdots, X_{t-w+1}) \approx {y_{t+1}}
\end{equation}
其中$\hat y$表示对$y$变量的预测值,非负整数$w$(window size)表示我们在第$t$天时,回顾历史(含当天)的时间跨度。在最简单的模型中,$w=1$,即
\begin{equation}
	\hat y_{t+1} = f(X_t) \approx {y_{t+1}}
\end{equation}

本项工作中我们主要采用平均绝对误差(Mean Absolute Error,MAE)来衡量模型$f$的预测精度。对于$n$条样本$y_1, y_2 \cdots, y_{n}$和模型对它们的预测值$\hat{y_1}, \hat{y_2}, \cdots, \hat{y_{n}}$,定义平均绝对误差$\epsilon_{MAE}$如下:
\begin{equation}
	\epsilon_{MAE} = \frac 1 n \sum_{i=1}^{n} | \hat{y_i} - y_i | 
\end{equation}







