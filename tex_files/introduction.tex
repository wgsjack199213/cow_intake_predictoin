\section{问题描述}
\label{introduction}


泌乳牛的草料采食量会影响其产奶量及其健康状况。
在实际生产中草料需要提前1天时间制备,并且1天后吃剩的草料将不再提供给牛食用,以免变质的草料影响牛的健康情况。因此,饲养员需要提前估计未来1天内牛的自由采食量,并由此上报需要制备的草料量。理想情况下,饲养员希望控制报草量(即实际发放至牛棚的草料量)略大于牛的自由采食量,即一天后草料有剩余但剩余不多,以免不能满足牛的草料需求或者造成草料的浪费。具体地,牧场规定草料剩余量最好不超过当天报草量的5\%。

泌乳牛的草料采食量受到多方面因素的影响,因而每天每牛棚的自由采食量会发生波动。传统生产中饲养员会根据经验对相关因素进行估计判断,以估计每天每牛棚的报草量。
这种方式可能会由于饲养员的经验差异导致或大或小的预测误差,可能会造成饲料不足或饲料浪费的问题。
本工作希望通过对历史采集数据的分析,量化地构建预测泌乳牛采食量的模型,从而给饲养员提供较准确的采食量预测作为参考,辅助饲养员更好地制定报草量。

本文分为以下几个部分:
第\ref{related}节简述关于采食量预测的相关工作情况。
第\ref{analyze}节对采食量预测问题进行分析并形式化地描述问题。
第\ref{dataset}节概述建模所使用的数据以及对数据的处理方式,
第\ref{model}节介绍预测模型,第\ref{evaluation}节展示实验结果并对结果进行分析。
第\ref{discussion}节讨论分析依据采食量预测值指定草料投放量面临的问题。
第\ref{futurework}节整理结论并讨论下一步的工作方向。
附录中第\ref{calving_data}节介绍泌乳天数和胎次数据如何获取,第\ref{concat_exp_data}节和第\ref{dwt_exp_data}节是相关实验更具体的结果数据。
